\documentclass{article}

%% Page Margins %%
\usepackage{geometry}
\geometry{
    top = 0.75in,
    bottom = 0.75in,
    right = 0.75in,
    left = 0.75in,
}

\usepackage{amsmath}
\usepackage{graphicx}
\usepackage{parskip}

\title{Assembly Project: Dr Mario}
\date{} 

\author{Rafay Usman and Sergio Sanchez}

\begin{document}
\maketitle
\section{Instruction and Summary}

\begin{enumerate}

    \item Which milestones were implemented? M1, M2, M3, Partial M4

    \item How to view the game:  
    \begin{enumerate}
    \item Unit width in pixels:       256
    \item Unit height in pixels:      256
    \item Display width in pixels:    8
    \item Display height in pixels:   8
    \item Base Address for Display:   0x10008000
    \end{enumerate}

    

\begin{figure}[ht!]
    \centering
    \includegraphics[width=0.3\textwidth]{mario game.png}
    \caption{caption}
    \label{Instructions}
\end{figure}



\begin{figure}[ht!]
    \centering
    \includegraphics[width=0.3\textwidth]{mario_gameover.png}
    \caption{caption}
    \label{Instructions}
\end{figure}


\item Game Summary:
% TODO: Tell  a little about your game.
\begin{itemize}
    \item Currently has droping blocks with increasing gravity space
    \item Collision Detection and block removal
    \item Game Over Screen
\end{itemize}

    
\end{enumerate}

\section{Attribution Table}
\begin{center}
\begin{tabular}{|| c | c ||}
\hline
 Student 1 (Rafay Usman and 1010103317) &  Student 2 (Sergio Sanchez 1008801432) \\ 
 \hline
 Block Removal Upon 4 in row and ensuring unsupported blocks fall (M3)& M1 \\
 \hline
 Gravity Speed & M2 \\
 \hline
 Deifficulty Selction (inprogress) & M3 \\ 
 \hline
 Added Pause & Gravity \\ 
 \hline
 Task & Game Over\\
 \hline
 Task & Task\\  
 \hline
\end{tabular}
\end{center}

% TODO: Fill out the remainder of the document as you see 
%       fit, including as much detail as you think 
%       necessary to better understand your code. 
%       You can add extra sections and subsections to 
%       help us understand why you deserve marks for 
%       features that were more challenging than they
%       might initially seem.


\end{document}